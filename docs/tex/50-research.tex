\chapter{Исследовательский раздел}

% В данном разделе приведено исследование влияния разработанного метода на характеристики выполнения программ на языке Golang, реализующих различные классы алгоритмов.

\section*{Цель исследования}

Реализованный метод распределения памяти может иметь различные сценарии использования и каждом из них оказывать различное влияние на характеристики выполнения программ. Предполагается, что изменение характеристик выполнения программы при применении реализованного метода зависит от класса алгоритма, который она реализует (см. п. \ref{alg_classes}).

Целью исследования является проведение сравнительного анализа разработанного метода с реализацией, существующей программирования Golang.

\section*{Описание исследования}

Для проведения сравнительного анализа необходимо установить зависимость между длиной входа алгоритма и временем его выполнения для различных классов алгоритмов и для двух методов распределения памяти: встроенного и реализованного. Для этого были подготовлены тесты производительности для следующих алгоритмов:

\begin{enumerate}[label*=\arabic*)]
	\item алгоритм конвейерной обработки данных (класс VC);
	\item алгоритм сортировки слиянием (класс VL);
	\item алгоритм нахождения расстояния Левенштейна (класс VQ);
	\item алгоритм умножения матриц по Винограду (класс VP);
	\item алгоритм сортировки подсчётом (класс VE).
\end{enumerate}

Для каждого алгоритма были разработаны две реализации: одна использует встроенные средства языка Golang, а другая --- реализованный метод.



\section*{Технические характеристики оборудования}

Ниже приведены технические характеристики устройства, на котором выполнялось тестирование.

\begin{itemize}[label*=---]
	\item Операционная система: Debian Linux 12 (bookworm) x86\_64, версия ядра 6.1.0-17.
	\item Объём оперативной памяти: 16 Гб.
	\item Процессор: Intel i5-9300H 2.4 ГГц.
\end{itemize}

Тестирование проводилось на ноутбуке, включенном в сеть электропитания. Во время тестирования ноутбук был нагружен только встроенными приложениями окружения, а также непосредственно системой тестирования.



\section*{Результаты исследования}

\subsection*{Алгоритм конвейерной обработки данных}

Для проведения замеров был разработан конвейер, состоящий из трёх линий: аллокация буфера фиксированного размера, запись данных в него и их чтение. В таблице \ref{tab:pipeline} представлены результаты замеров времени выполнения алгоритма при различном количестве заявок. На рисунке \ref{fig:time-pipeline} представлены данные из таблицы \ref{tab:pipeline} в виде столбчатой диаграммы.

\captionsetup{format=hang,justification=raggedright, singlelinecheck=off}
\begin{table}[H]
	\centering
	\caption{Замеры времени выполнения алгоритма конвейерной обработки данных}\label{tab:pipeline}
	\renewcommand{\arraystretch}{1.2}
	\begin{tabular}{|J|K|K|}
		\hline
		\multicolumn{1}{|I|}{\textit{Число заявок, шт}} & \multicolumn{1}{G}{\textit{Время выполнения при использовании реализованного метода, мс}} & \multicolumn{1}{|G|}{\textit{Время выполнения при использовании встроенных средств языка, мс}} \\ \hline
		10 & 271 & 321 \\ \hline
		50 & 476 & 458 \\ \hline
		100 & 702 & 678 \\ \hline
		500 & 1811 & 1720 \\ \hline
		1000 & 3389 & 3248 \\ \hline
		3000 & 11673 & 9290 \\ \hline
		5000 & 19897 & 15272 \\ \hline
		7500 & 28661 & 21387 \\ \hline
		10000 & 37349 & 27465 \\ \hline
	\end{tabular}
\end{table}

\begin{figure}[H]
	\centering
	\begin{tikzpicture}
		\begin{semilogyaxis} [
			ybar,
			xlabel={Число заявок, шт},
			ylabel={Время выполнения алгоритма, мс},
			grid = major,
			grid style = {dashed, lightgray!35},
			xtick={0, 1, 2, 3, 4, 5, 6, 7, 8},
			xticklabels={$10$, $50$, $100$, $500$, $1000$, $3000$, $5000$, $7500$, $10000$},
			ytick={10^2.3, 10^2.9, 10^3.5, 10^4.1, 10^4.7},
			bar width = 12pt,
			width = 0.95\textwidth,
			height = 0.66\textwidth,
			tick label style={font=\scriptsize},
			scaled ticks=false,
			legend entries={Реализованный метод, Встроенные средства языка},
			legend pos=north west,
			legend image post style={scale=1.9}
			]
			
			\addplot [
			draw = blue,
			semithick,
			pattern = vertical lines,
			pattern color = blue
			]   coordinates {
				(0, 271)
				(1, 476)
				(2, 702)
				(3, 1811)
				(4, 3389)
				(5, 11673)
				(6, 19897)
				(7, 28661)
				(8, 37349)
			};
			
			\addplot [
			draw = red,
			semithick,
			pattern = horizontal lines,
			pattern color = red
			] coordinates {
				(0, 321)
				(1, 458)
				(2, 678)
				(3, 1720)
				(4, 3248)
				(5, 9290)
				(6, 15272)
				(7, 21387)
				(8, 27465)
			};
		\end{semilogyaxis}
	\end{tikzpicture}
	\caption{Зависимость времени выполнения алгоритма конвейерной обработки данных от числа заявок}
	\label{fig:time-pipeline}
\end{figure}

\subsection*{Алгоритм сортировки слиянием}

В таблице \ref{tab:mergesort} представлены результаты замеров времени выполнения алгоритма сортировки слиянием при различных размерах входного массива целых чисел. На рисунке \ref{fig:time-mergesort} представлены данные из таблицы \ref{tab:mergesort} в виде столбчатой диаграммы.

\begin{table}[H]
	\centering
	\caption{Замеры времени выполнения алгоритма сортировки слиянием}\label{tab:mergesort}
	\renewcommand{\arraystretch}{1.2}
	\begin{tabular}{|J|K|K|}
		\hline
		\multicolumn{1}{|I|}{\textit{Число элементов в массиве, шт}} & \multicolumn{1}{G}{\textit{Время выполнения при использовании реализованного метода, мс}} & \multicolumn{1}{|G|}{\textit{Время выполнения при использовании встроенных средств языка, мс}} \\ \hline
		100 & 91 & 13 \\ \hline
		500 & 694 & 111 \\ \hline
		1000 & 993 & 241 \\ \hline
		5000 & 5138 & 1566 \\ \hline
		10000 & 10037 & 3855 \\ \hline
		30000 & 30771 & 10163 \\ \hline
		50000 & 51519 & 16459 \\ \hline
		75000 & 76830 & 29381 \\ \hline
		100000 & 101879 & 42221 \\ \hline
	\end{tabular}
\end{table}

\begin{figure}[H]
	\centering
	\begin{tikzpicture}
		\begin{semilogyaxis} [
			ybar,
			xlabel={Число элементов в массиве, шт},
			ylabel={Время выполнения алгоритма, мс},
			grid = major,
			grid style = {dashed, lightgray!35},
			xtick={0, 1, 2, 3, 4, 5, 6, 7, 8},
			xticklabels={$100$, $500$, $1000$, $10000$, $30000$, $50000$, $75000$, $10000$, $100000$},
			bar width = 12pt,
			width = 0.95\textwidth,
			height = 0.66\textwidth,
			tick label style={font=\scriptsize},
			scaled ticks=false,
			legend entries={Реализованный метод, Встроенные средства языка},
			legend pos=north west,
			legend image post style={scale=1.9}
			]
			
			\addplot [
			draw = blue,
			semithick,
			pattern = vertical lines,
			pattern color = blue
			]   coordinates {
				(0, 91)
				(1, 694)
				(2, 993)
				(3, 5138)
				(4, 10037)
				(5, 30771)
				(6, 51519)
				(7, 76830)
				(8, 101879)
			};
			
			\addplot [
			draw = red,
			semithick,
			pattern = horizontal lines,
			pattern color = red
			] coordinates {
				(0, 13)
				(1, 111)
				(2, 241)
				(3, 1566)
				(4, 3855)
				(5, 10163)
				(6, 16459)
				(7, 29381)
				(8, 42221)
			};
		\end{semilogyaxis}
	\end{tikzpicture}
	\caption{Зависимость времени выполнения алгоритма сортировки слиянием от числа элементов в массиве}
	\label{fig:time-mergesort}
\end{figure}

\subsection*{Алгоритм нахождения расстояния Левенштейна}

В таблице \ref{tab:levenstein} представлены результаты замеров времени выполнения нерекурсивного алгоритма нахождения расстояния Левенштейна между строками равной длины.

\begin{table}[H]
	\centering
	\caption{Замеры времени выполнения алгоритма нахождения расстояния Левенштейна}\label{tab:levenstein}
	\renewcommand{\arraystretch}{1.2}
	\begin{tabular}{|J|K|K|}
		\hline
		\multicolumn{1}{|I|}{\textit{Число символов в строках, шт}} & \multicolumn{1}{G}{\textit{Время выполнения при использовании реализованного метода, мс}} & \multicolumn{1}{|G|}{\textit{Время выполнения при использовании встроенных средств языка, мс}} \\ \hline
		100 & 184 & 42 \\ \hline
		300 & 1780 & 326 \\ \hline
		500 & 5258 & 573 \\ \hline
		750 & 11621 & 1818 \\ \hline
		1000 & 19222 & 3483 \\ \hline
		2000 & 73641 & 13963 \\ \hline
		3000 & 166382 & 31616 \\ \hline
		4000 & 346885 & 53427 \\ \hline
		5000 & 544065 & 86802 \\ \hline
	\end{tabular}
\end{table}

\clearpage
На рисунке \ref{fig:time-levenstein} представлены данные из таблицы \ref{tab:levenstein} в виде столбчатой диаграммы.

\begin{figure}[H]
	\centering
	\begin{tikzpicture}
		\begin{semilogyaxis} [
			ybar,
			xlabel={Число символов в строках, шт},
			ylabel={Время выполнения алгоритма, мс},
			grid = major,
			grid style = {dashed, lightgray!35},
			xtick={0, 1, 2, 3, 4, 5, 6, 7, 8},
			xticklabels={$100$, $300$, $500$, $750$, $1000$, $2000$, $3000$, $4000$, $5000$},
			bar width = 12pt,
			width = 0.95\textwidth,
			height = 0.66\textwidth,
			tick label style={font=\scriptsize},
			scaled ticks=false,
			legend entries={Реализованный метод, Встроенные средства языка},
			legend pos=north west,
			legend image post style={scale=1.9}
			]
			
			\addplot [
			draw = blue,
			semithick,
			pattern = vertical lines,
			pattern color = blue
			]   coordinates {
				(0, 184)
				(1, 1780)
				(2, 5258)
				(3, 11621)
				(4, 19222)
				(5, 73641)
				(6, 166382)
				(7, 346885)
				(8, 544065)
			};
			
			\addplot [
			draw = red,
			semithick,
			pattern = horizontal lines,
			pattern color = red
			] coordinates {
				(0, 42)
				(1, 326)
				(2, 573)
				(3, 1818)
				(4, 3483)
				(5, 13963)
				(6, 31616)
				(7, 53427)
				(8, 86802)
			};
		\end{semilogyaxis}
	\end{tikzpicture}
	\caption{Зависимость времени выполнения алгоритма нахождения расстояния Левенштейна от числа символов в строках}
	\label{fig:time-levenstein}
\end{figure}

\subsection*{Алгоритм умножения матриц по Винограду}

В таблице \ref{tab:winograd} представлены результаты замеров времени выполнения алгоритма Винограда умножения квадратных матриц равной размерности. На рисунке \ref{fig:time-winograd} представлены данные из таблицы \ref{tab:winograd} в виде столбчатой диаграммы.

\begin{table}[H]
	\centering
	\caption{Замеры времени выполнения алгоритма умножения матриц по Винограду}\label{tab:winograd}
	\renewcommand{\arraystretch}{1.2}
	\begin{tabular}{|J|K|K|}
		\hline
		\multicolumn{1}{|I|}{\textit{Размерность матриц}} & \multicolumn{1}{G}{\textit{Время выполнения при использовании реализованного метода, мс}} & \multicolumn{1}{|G|}{\textit{Время выполнения при использовании встроенных средств языка, мс}} \\ \hline
		10 & 8 & 8 \\ \hline
		50 & 222 & 303 \\ \hline
		100 & 1495 & 2196 \\ \hline
		300 & 52627 & 78902 \\ \hline
		500 & 227228 & 406249 \\ \hline
		650 & 462251 & 839183 \\ \hline
		750 & 838280 & 1401997 \\ \hline
		900 & 1508186 & 3053736 \\ \hline
		1000 & 3049303 & 7017908 \\ \hline
	\end{tabular}
\end{table}

\begin{figure}[H]
	\centering
	\begin{tikzpicture}
		\begin{semilogyaxis} [
			ybar,
			xlabel={Размерность матриц},
			ylabel={Время выполнения алгоритма, мс},
			grid = major,
			grid style = {dashed, lightgray!35},
			xtick={0, 1, 2, 3, 4, 5, 6, 7, 8},
			xticklabels={$10$, $50$, $100$, $300$, $500$, $650$, $750$, $900$, $1000$},
			bar width = 12pt,
			width = 0.95\textwidth,
			height = 0.66\textwidth,
			tick label style={font=\scriptsize},
			scaled ticks=false,
			legend entries={Реализованный метод, Встроенные средства языка},
			legend pos=north west,
			legend image post style={scale=1.9}
			]
			
			\addplot [
			draw = blue,
			semithick,
			pattern = vertical lines,
			pattern color = blue
			]   coordinates {
				(0, 7.742)
				(1, 222)
				(2, 1495)
				(3, 52627)
				(4, 227228)
				(5, 462251)
				(6, 838280)
				(7, 1508186)
				(8, 3049303)
			};
			
			\addplot [
			draw = red,
			semithick,
			pattern = horizontal lines,
			pattern color = red
			] coordinates {
				(0, 7.765)
				(1, 303)
				(2, 2196)
				(3, 78902)
				(4, 406249)
				(5, 839183)
				(6, 1401997)
				(7, 3053736)
				(8, 7017908)
			};
		\end{semilogyaxis}
	\end{tikzpicture}
	\caption{Зависимость времени выполнения алгоритма умножения матриц по Винограду от их размерности}
	\label{fig:time-winograd}
\end{figure}

\subsection*{Алгоритм сортировки подсчётом}

В таблице \ref{tab:countingsort} представлены результаты замеров времени выполнения алгоритма сортировки подсчётом при различных размерах входного массива целых чисел, находящихся в полуинтервале $[0;10^7)$. На рисунке \ref{fig:time-countingsort} представлены данные из таблицы \ref{tab:countingsort} в виде столбчатой диаграммы.

\begin{table}[H]
	\centering
	\caption{Замеры времени выполнения алгоритма сортировки подсчётом}\label{tab:countingsort}
	\renewcommand{\arraystretch}{1.2}
	\begin{tabular}{|J|K|K|}
		\hline
		\multicolumn{1}{|I|}{\textit{Число элементов в массиве, шт}} & \multicolumn{1}{G}{\textit{Время выполнения при использовании реализованного метода, мс}} & \multicolumn{1}{|G|}{\textit{Время выполнения при использовании встроенных средств языка, мс}} \\ \hline
		1000 & 14687 & 18527 \\ \hline
		10000 & 15220 & 18667 \\ \hline
		100000 & 16556 & 20360 \\ \hline
		500000 & 20712 & 23525 \\ \hline
		1000000 & 28030 & 32345 \\ \hline
		3000000 & 52895 & 55984 \\ \hline
		5000000 & 78564 & 78069 \\ \hline
		7500000 & 110101 & 109593 \\ \hline
		10000000 & 138367 & 137969 \\ \hline
	\end{tabular}
\end{table}

\begin{figure}[H]
	\centering
	\begin{tikzpicture}
		\begin{semilogyaxis} [
			ybar,
			xlabel={Размерность задачи},
			ylabel={Время выполнения алгоритма, мс},
			grid = major,
			grid style = {dashed, lightgray!35},
			xtick={0, 1, 2, 3, 4, 5, 6, 7, 8},
			xticklabels={$1\cdot10^3$, $1\cdot10^4$, $1\cdot10^5$, $5\cdot10^5$, $1\cdot10^6$, $3\cdot10^6$, $5\cdot10^6$, $7.5\cdot10^6$, $1\cdot10^7$},
			bar width = 14pt,
			width = 0.95\textwidth,
			height = 0.66\textwidth,
			tick label style={font=\scriptsize},
			scaled ticks=false,
			legend entries={Реализованный метод, Встроенные средства языка},
			legend pos=north west,
			legend image post style={scale=1.9}
			]
			
			\addplot [
			draw = blue,
			semithick,
			pattern = vertical lines,
			pattern color = blue
			]   coordinates {
				(0, 14687)
				(1, 15220)
				(2, 16556)
				(3, 20712)
				(4, 28030)
				(5, 52895)
				(6, 78564)
				(7, 110101)
				(8, 138367)
			};
			
			\addplot [
			draw = red,
			semithick,
			pattern = horizontal lines,
			pattern color = red
			] coordinates {
				(0, 18527)
				(1, 18667)
				(2, 20360)
				(3, 23525)
				(4, 32345)
				(5, 55984)
				(6, 78069)
				(7, 109593)
				(8, 137969)
			};
		\end{semilogyaxis}
	\end{tikzpicture}
	\caption{Зависимость времени выполнения алгоритма сортировки подсчётом от числа элементов в массиве}
	\label{fig:time-countingsort}
\end{figure}



\section*{Выводы из исследовательского раздела}

В данном разделе было проведено исследование влияния разработанного метода распределения памяти на время выполнения программ, реализующих различные классы алгоритмов. Сравнительный анализ показал, что для классов алгоритмов VP и VE разработанный метод оказался эффективнее реализации, существующей в языке программирования Golang. Использование разработанного метода позволило снизить время выполнения алгоритма Винограда умножения матриц (класс VP) на 0.3-56.5\% в зависимости от размерности матриц, а время выполнения алгоритма сортировки подсчётом (класс VE) на 5.8-20.7\% в зависимости от размера входного массива при условии, что он не достигает 5000000.

