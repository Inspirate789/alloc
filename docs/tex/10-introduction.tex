\part*{ВВЕДЕНИЕ}
\addcontentsline{toc}{part}{\textbf{ВВЕДЕНИЕ}}

%Во введении написать:
%Объект исследования - работа с памятью в языках программирования
%Предмет исследования - методы распределения памяти в современных языках программирования
%Гипотеза - эффективность работы с памятью в языках программирования можно повысить на основе новых качеств методов распределения памяти (методов автоматического управления памятью)
%
%Актуальность - современные приложения, в особенности разработанные для длительной непрерывной работы, подвержены следующим факторам
%Утечки памяти
%Фрагментация памяти
%Длительные паузы на сборку мусора
%Перечисленные факторы могут негативно сказываться на производительности и отказоустойчивости приложений.
%
%Поэтому (((гипотеза))). Для этого необходимо
%1) выделить основные свойства существующих методов распределения памяти в современных языках программирования
%2) разработать метод распределения памяти, опирающийся на свойства существующих решений
%3) определить границы применимости разработанного метода
%
%Цель - разработка метода ...
%Задачи...
%
%И уже в постановке задачи и построении IDEF0 сделать вывод из двух классификаций, сказав, что у существующих решений есть недостатки и можно разработать свой метод, нацеленный на устранение таких-то недостатков методов (или более конкретно: такого недостатка у метода в ЯП1, другого у метода ЯП2 и т.д.). При этом разработанный метод должен быть ориентирован на применение в реализациях алгоритмов таких-то классов, так как именно они требуют выделения объёма памяти в куче, пропорционального входу алгоритма
%
%Методология исследования:
%?????



Память является одним из основных ресурсов любой вычислительной системы, требующих тщательного управления. Под памятью (memory) в работе будет подразумеваться оперативная память компьютера. Особая роль памяти объясняется тем, что процессор может выполнять инструкции программы только в том случае, если они находятся в памяти. Память распределяется между операционной системой компьютера и прикладными программами.~\cite{tannenbaum}

Для выполнения вычислений в языках программирования используются объекты, которые могут быть представлены как простым типом данных (целые числа, символы, логические значения и т.д.) так и агрегированным (массивы, списки, деревья и т.д.). Значения объектов программ хранятся в памяти для быстрого доступа. Во многих языках программирования переменная в программном коде --- это адрес объекта в памяти.~\cite{c}~\cite{cpp}~\cite{golang} Когда переменная используется в программе, процесс считывает значение из памяти и обрабатывает его.

Большинство современных языков программирования использует динамическое распределение памяти, при котором выделение объектов осуществляется во время выполнения программы. Динамическое управление памятью вводит два основных примитива --- функции выделения и освобождения памяти, за которые отвечает \textbf{аллокатор}. 

Существует два способа управления динамической памятью --- ручное и автоматическое. При ручном управлении памятью программист должен следить за освобождением выделенной памяти, что приводит к возможности возникновения ошибок. Более того, в некоторых ситуациях (например, при программировании на функциональных языках или в многопоточной среде) время жизни объекта не всегда очевидно для разработчика.~\cite{elixir} Автоматическое управление памятью избавляет программиста от необходимости вручную освобождать выделенную память, устраняя тем самым целый класс возможных ошибок и увеличивая безопасность разрабатываемых программ. Сборка мусора (garbage collection) за последние два десятилетия стала стандартом в области автоматического управления памятью, хотя её использование накладывает дополнительные расходы памяти и времени исполнения. На сегодняшний день среды времени выполнения (language runtime) многих популярных языков программирования, таких как Java, C\#, Python и другие, активно используют сборку мусора. 

Современные приложения, в особенности разработанные для длительной непрерывной работы, подвержены влиянию следующих факторов: 
\begin{enumerate}[label*=\arabic*)]
	\item утечки памяти;
	\item фрагментация памяти;
	\item длительные паузы на сборку мусора.
\end{enumerate}
Перечисленные факторы могут негативно сказываться на производительности и отказоустойчивости приложений. Снизить влияние данных факторов на работу программ и повысить эффективность работы с памятью можно на основе новых качеств методов автоматического управления памятью.

Целью данной работы является разработка метода автоматического управления памятью с гарантированным временем выполнения на основе подсчёта ссылок. Для достижения поставленной цели необходимо решить следующие задачи.

\begin{enumerate}[label*=\arabic*.]
	\item Провести анализ и классификацию существующих методов распределения памяти в языках программирования с автоматической сборкой мусора.
	\item Провести классификацию алгоритмов для определения области применения разрабатываемого метода.
	\item Спроектировать и реализовать метод автоматического управления памятью.
	\item Определить границы применения реализованного метода. 
\end{enumerate}
