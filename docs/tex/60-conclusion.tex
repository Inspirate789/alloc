\part*{ЗАКЛЮЧЕНИЕ}
\addcontentsline{toc}{part}{\textbf{ЗАКЛЮЧЕНИЕ}} 

В рамках настоящей работы был разработан и реализован метод распределения памяти. Все поставленные задачи были выполнены.

Был проведён анализ предметной области автоматического управления памятью, а также классификация существующих менеджеров памяти языков программирования. На основе их классификации были сформулированы концепции комбинированного метода: минимизация накладных расходов мутатора, времени сбора мусора и объёма памяти для хранения данных об объектах, применение конкурентной и параллельной сборки мусора, а также использование алгоритма поколений.

Была проведена классификация компьютерных алгоритмов по требованиям к дополнительной памяти. Для дальнейшего рассмотрения были выбраны алгоритмы, относящиеся к классам VC, VL, VQ, VP и VE.

Была дана гарантия времени выполнения реализованного метода в виде теоретической оценки трудоёмкости выполнения выделения памяти и обращения к выделенным объектам. Благодаря использованию модели поколений, удалось устранить зависимость накладных расходов мутатора от общего числа объектов во всей куче.

Для классов алгоритмов VP и VE разработанный метод оказался эффективнее реализации, существующей в языке программирования Golang. Использование разработанного метода позволило снизить время выполнения алгоритма Винограда умножения матриц (класс VP) на 0.3-56.5\% в зависимости от размерности матриц, а время выполнения алгоритма сортировки подсчётом (класс VE) на 12-20.7\% в зависимости от размера входного массива при условии, что он не достигает 5000000.

В качестве дальнейшего развития предлагается внедрение разработанного метода или отдельных его концепций в основной сборщик мусора языка Golang. Также в дальнейшем планируется исследование стабильности разработанного менеджера памяти, а также фрагментации кучи при его работе.
