\part*{ОПРЕДЕЛЕНИЯ, ОБОЗНАЧЕНИЯ И\\СОКРАЩЕНИЯ}
\addcontentsline{toc}{part}{\textbf{ОПРЕДЕЛЕНИЯ, ОБОЗНАЧЕНИЯ И СОКРАЩЕНИЯ}}
В настоящей расчетно-пояснительной записке применяют следующие термины с соответствующими определениями.

\begin{enumdescript}
	\item[Стек] --- область адресного пространства процесса, предназначенная для хранения параметров функций, локальных переменных и адреса возврата после вызова функции.
	\item[Куча] --- область адресного пространства процесса, предназначенная для выделения памяти, динамически запрашиваемой программой.
	\item[Внутренняя фрагментация] --- явление, при котором аллокатор выделяет при каждом запросе больше памяти, чем фактически запрошено.
	\item[Внешняя фрагментация] --- явление, при котором свободная память разделена на множество мелких блоков, ни один из которых нельзя использовать для обслуживания запроса на выделение памяти.
	\item[Управление памятью] --- процесс координации и контроля использования памяти в вычислительной системе.
	\item[Автоматическое управление памятью] --- служба, которая автоматически перерабатывает память, которую программа не собирается использовать в дальнейшем
	\item[Мусор] --- объект, о котором достоверно известно, что он не используется программой, так как является недостижимым\footnote{Термин <<недостижимость>> может иметь различные интерпретации в разных языках программирования} из других её объектов.
	\item[Сбор мусора] --- автоматическая переработка динамически выделяемой памяти.
	\item[Сборщик мусора] --- автоматический менеджер памяти, реализующий некоторый алгоритм сборки мусора.
	\item[Мутатор] --- пользовательская программа, которая изменяет граф объектов.
	\item[Барьер] --- функция, которая принимает указатель на объект в памяти, анализирует его статус и в зависимости от него выполняет какие-либо действия с этим указателем или даже с объектом, на который он ссылается, после чего возвращает актуальное значение указателя, которое можно использовать для доступа к объекту.
	\item[Поколение] --- группа объектов со схожим временем жизни.
	\item[Финализатор объекта] --- функция, которая запускается, когда сборщик мусора определяет, что рассматриваемый объект недоступен в программе.
	\item[Финализация объекта] --- процесс выполнения финализаторов объекта.
\end{enumdescript}
