\part*{РЕФЕРАТ}
%\thispagestyle{empty}
\addcontentsline{toc}{part}{\textbf{РЕФЕРАТ}}

Расчетно--пояснительная записка \pageref{LastPage} с., \totalfigures\ рис., \totaltables\ табл., \thetotalbibentries\ ист., 1 прил.

Объектом исследования является работа с памятью в языках программирования. Цель работы --- спроектировать и реализовать метод автоматического управления памятью с гарантированным временем выполнения на основе подсчёта ссылок.

Реализованный менеджер памяти основан на модели поколений и использует конкурентную и параллельную сборку мусора. Использование разработанного метода позволило снизить время выполнения алгоритмов, относящихся к классам VP и VE, на 0.3-56.5\% и 12-20.7\% соответственно в зависимости от длины входа.

Применение новых методов автоматического управления памятью может повысить эффективность приложений, в особенности разработанных для длительной непрерывной работы.

%УПРАВЛЕНИЕ ПАМЯТЬЮ, СБОРКА МУСОРА, JAVA, PYTHON
%
%Цель работы: классификация алгоритмов распределения памяти в языках программирования с автоматической сборкой мусора.
%
%В данной работе проводится обзор и сравнение алгоритмов автоматического управления памятью в языках программирования Python, Java, JavaScript, C\# и Golang.
%
%Среди рассмотренных автоматических менеджеров памяти были выделены менеджеры памяти языков программирования Python и Java как наиболее масштабируемые и оптимизированные для различных сценариев использования.
